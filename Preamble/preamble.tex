% ******************************************************************************
% ******************************* Packages *************************************
% ******************************************************************************

% ****************************** Bibliography **********************************
% https://de.sharelatex.com/learn/biblatex_bibliography_styles
% https://tex.stackexchange.com/questions/25701/bibtex-vs-biber-and-biblatex-vs-natbib

% \usepackage{natbib}
% \usepackage[numbers,compress]{natbib}
% \usepackage[backend=bibtex, style=numeric, citestyle=numeric, sorting=none, natbib=true,url=false]{biblatex}
% \usepackage[backend=bibtex, style=numeric-comp, citestyle=numeric, sorting=none, natbib=true,url=false]{biblatex}

\usepackage[backend=bibtex, mincitenames=1, maxcitenames=3, style=ieee, citestyle=numeric, sorting=none, natbib=true, doi=false, isbn=false, url=false]{biblatex}

% **************************** Hyphenation ************************************

\ifenglish
    \usepackage[british]{babel} 
\else
    \usepackage[portuges, brazilian]{babel} 
\fi

% *********************************** SI Units *********************************
\usepackage{siunitx} % use this package module for SI units

% ******************* Sections, Table of Contents & Appendices *****************
% add Bibliography, List of figures and tables to contents
\usepackage{epigraph}
\usepackage{footnote}
\usepackage{tocbibind}
\usepackage[title,titletoc]{appendix} % Add appendices

% ******************************* Nomenclature *********************************
\usepackage[nopostdot,nonumberlist,acronyms,section]{glossaries}

% *************************** Graphics and Figures *****************************
\usepackage{varwidth}
\usepackage{layout}
\usepackage{caption}
\usepackage{graphics}
\usepackage{tikz}
\usepackage{subcaption}
\usepackage{graphicx}
\ifpdf
\usepackage{epstopdf} % Convert eps figures to pdf
\fi
\usepackage[section]{placeins}  % Ensure that the figures be placed in the same section

% \usepackage[normalsize]{subfigure} 
% \usepackage[usenames, dvipsnames]{color}
% \usepackage[outline]{contour}
% \usepackage{psfrag}
%\usepackage{rotating}
%\usepackage{wrapfig}

% Uncomment the following two lines to force Latex to place the figure.
% Use [H] when including graphics. Note 'H' instead of 'h'
%\usepackage{float}
%\restylefloat{figure}

% ***************************** Tables Packages ********************************
\usepackage{booktabs} % For professional looking tables
\usepackage{tabularx,tabulary}
\usepackage{multirow}
% \usepackage{bookmark}
% \usepackage{array}
%\usepackage{multicol}
%\usepackage{longtable}

% ************************* Algorithms and Pseudocode **************************
%\usepackage{algpseudocode}

% *********************** Meta Data PDF ********************************
\usepackage[unicode=true, linktocpage=true]{hyperref}

% ******************************* Others *********************************
\usepackage{needspace} % Break the page if there is a certain ammount of space below


% ******************************************************************************
% ******************************* Options **************************************
% ******************************************************************************

% ******************************* Line Spacing *********************************
% Choose linespacing as appropriate. Default is one-half line spacing as per the
% FEM guidelines

% \doublespacing
\onehalfspacing
% \singlespacing

% ********************** TOC depth and numbering depth *************************
\setcounter{secnumdepth}{2}
\setcounter{tocdepth}{2}

% ******************************* References ***********************************
\addbibresource{References/references.bib}  % References path
% \addbibresource{References/references.bib}
% \renewcommand{\bibname}{References}
% \renewcommand{\refname}{References}


% ********************************* Appendix ***********************************
% The default value of both \appendixtocname and \appendixpagename is `Appendices'. These names can all be changed via:
%\renewcommand{\appendixtocname}{List of appendices}
\renewcommand{\appendixname}{Appendix}

% *************************** Graphics and Figures *****************************
\usetikzlibrary{shapes,arrows}
\usetikzlibrary{positioning,calc}

\ifpdf
  \DeclareGraphicsExtensions{.png, .jpg, .pdf}
  \graphicspath{{Figs/raster/}{Figs/}}
\else
  \DeclareGraphicsExtensions{.eps, .ps}
  \graphicspath{{Figs/vector/}{Figs/}}
\fi

% ******************************* Nomenclature *********************************
% The list of symbols are a supergroup defined below
\newglossarystyle{supergroup}{%
  \renewenvironment{theglossary}%
  {\tablehead{}\tabletail{}%
   \begin{supertabular}{@{}l@{}lp{\glsdescwidth}}}%
  {\end{supertabular}}%
  \renewcommand*{\glossaryheader}{}%
  \renewcommand*{\glsgroupskip}{}%
  \renewcommand*{\glossentry}[2]{%
  \tabularnewline
    &\multicolumn{2}{@{}l}{%
     \bfseries\glsentryitem{##1}\glstarget{##1}{\glossentryname{##1}}%
    }% 
    \tabularnewline
  }%
  \renewcommand{\subglossentry}[3]{%
     &\glssubentryitem{##2}%
     \glstarget{##2}{\glossentryname{##2}}
     &
     \glossentrydesc{##2}\glspostdescription\space
     ##3%
     \tabularnewline
  }%
}

\renewcommand{\glossarysection}[2][]{}
% \ifenglish
  % Create the glossary Symbols
  \newglossary[slg]{symbol}{sot}{stn}{Symbols}
  % Define the subgroups
  \newglossaryentry{alpha}{type=symbol, name={Alphanumeric}, description={}}
  \newglossaryentry{greek}{type=symbol, name={Greek Symbols}, description={}}
  \newglossaryentry{roman}{type=symbol, name={Roman Symbols}, description={}}
  \newglossaryentry{super}{type=symbol, name={Superscripts}, description={}}
  \newglossaryentry{sub}{type=symbol, name={Subscripts}, description={}}
  \newglossaryentry{osymb}{type=symbol, name={Other Symbols}, description={}}
% \else
%   % Create the glossary Symbols
%   \newglossary[slg]{symbol}{sot}{stn}{Símbolos}
%   % Define the subgroups
%   \newglossaryentry{alpha}{type=symbol, name={Alfanumérico}, description={}}
%   \newglossaryentry{greek}{type=symbol, name={Letras Gregas}, description={}}
%   \newglossaryentry{roman}{type=symbol, name={Letras Romanas}, description={}}
%   \newglossaryentry{super}{type=symbol, name={Superescritos}, description={}}
%   \newglossaryentry{sub}{type=symbol, name={Subescritos}, description={}}
%   \newglossaryentry{osymb}{type=symbol, name={Outros Símbolos}, description={}}
% \fi
\makeglossaries

% *********************** Meta Data PDF ********************************
\ifprint
  % For Print version
  \hypersetup{
    final=true,
    plainpages=false,
    pdfstartview=FitV,
    pdftoolbar=true,
    pdfmenubar=true,
    bookmarksopen=true,
    bookmarksnumbered=true,
    breaklinks=true,
    linktocpage,
    colorlinks=true,
    linkcolor=black,
    urlcolor=black,
    citecolor=black,
    anchorcolor=black
  }
\else
  % For PDF Online version
  \hypersetup{
    final=true,
    plainpages=false,
    pdfstartview=FitV,
    pdftoolbar=true,
    pdfmenubar=true,
    bookmarksopen=true,
    bookmarksnumbered=true,
    breaklinks=true,
    linktocpage,
    colorlinks=true,
    linkcolor=blue,
    urlcolor=blue,
    citecolor=blue,
    anchorcolor=green
  }
\fi

\hypersetup{
    pdftitle = {\ifenglish \Wtitleen \else \Wtitle \fi},
    pdfauthor = {\Wauthor},
    pdfsubject= {\Wsubject},
    pdfkeywords= {\Wkeywords}
}